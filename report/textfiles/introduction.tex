\section{Introduction}
Attitude maneuvers are a core problem to the control of aerospace vehicles. As such, understanding and optimizing the problem is a key aspect to the design of better performing vehicles. The performance index to be optimized will depend on the specific application and vehicle considered. For high performance aircraft and satellites, a possible performance index is the time used for a rest-to-rest maneuver. This has direct applications to satellite re-targeting or high-acceleration maneuvers for aircraft allowing for faster redirecting of the aircraft path. 

A survey on time-optimal maneuvers is provided by Scrivener \cite{scrivener1994survey}. Optimal maneuvers for different configurations: number of angular velocity controls, rigid and flexible bodies, and rotating around the eigenaxis is provided.

Here, since we aim to achieve insight on maneuvers for both spacecraft and aircraft we will focus on rigid-body maneuvers under bounded controls. For large angle reorientations introducing quaternions will simplify the problem of computing the difference between attitudes, besides avoiding the singularities of euler angles, as proposed by Wie and Weiss \cite{wie1989quaternion}. In this case they introduce a regulator that follows the eigenaxis rotation between initial and final state which is the minimum action path. Chowdhry \cite{chowdhry1991optimal} solves the time-optimization of the attitude stabilization (i.e. drive angular velocity to 0) for different configurations where one angular rate control is not available for its application to super-maneuverable aircraft.

The problem is interesting from a theoretical point of view since it involves a number of different tools. The attitude equations of motion are highly nonlinear, and without any assumptions result in complex motion. Attitude itself calls for the use of quaternions or similar formulations since Euler angle strategies may become badly-defined and little intuitive. Finally, optimal control theory is necessary in order to achieve the necessary conditions for the time optimality of the control as stated by Longuski \cite{longuski2014optimal}. Besides, the final control is not readily intuitive: eigenaxis maneuvers provide the minimum length path, however aircraft tend to combine rolling and pitching motion.

In particular, we will follow the approach used by Bilimoria \cite{bilimoria1993time} and Fleming \cite{fleming2010minimum} but using a slightly different notation. An axis-symmetric rigid-body will be considered where all three angular velocities can be controlled independently and where the control is norm by some norm. It would also be interesting to follow some of the approaches used by Byers \cite{byers1993quasi} to look for real-time solving capabilities for the control so that it could be flyable.


